\begin{frame}{Ludologia a narratologia}
Pierwszym krokiem do zrozumienia problemu ludologii i narracji w grach komputerowych jest wyjaśnienie dwóch podstawowych terminów: ludologii i narratologii.

Ludologia jest nauką o grach; dotyczy mechanizmów gry, analizy zasad i sposobu, w jaki gracz wchodzi z nią w interakcję. Z ludologicznego punktu widzenia ważne jest, aby zwrócić uwagę na cel gry, jej zasady, wyzwania i nagrody, sposób, w jaki gracz ingeruje w jej przebieg.

Z kolei narratologia to dziedzina badań zajmująca się teorią narracji lub sposobem, w jaki opowiadane są historie. W przypadku gier komputerowych narratologia bada, w jaki sposób wirtualne światy, postacie, fabuła i dialogi współgrają ze sobą, tworząc interesującą historię.
\end{frame}
