\begin{frame}{Wyzwania i kontrowersje}
Problem ludologiczno-narracyjny jest nie tylko teoretyczny, ale i praktyczny. Twórcy gier muszą zmierzyć się z wieloma trudnościami, takimi jak:

Problem wyboru gracza: swoboda podejmowania decyzji może sprawić, że fabuła będzie chaotyczna i niekontrolowana, podczas gdy zbyt mała interaktywność może sprawić, że gra będzie zbyt liniowa i przewidywalna.

Zgodność narracji z mechaniką: Czasami mechanika gry stoi w sprzeczności z fabułą. Na przykład w niektórych grach narracyjnych mechanika, która wymaga powtarzalnych manipulacji, może zniszczyć immersję i przenieść uwagę na same zasady gry, a nie na opowiadaną historię.

Jest to kwestia percepcji: nie wszyscy gracze preferują gry silnie narracyjne. Niektórzy gracze uważają, że interaktywność i mechanika są najważniejsze, a fabuła jest dla nich tylko dodatkiem. Inni gracze chcą uzyskać głębokie emocjonalne doświadczenie narracyjne i zniechęca ich zbyt złożona lub zbyt wymagająca mechanika.
\end{frame}
