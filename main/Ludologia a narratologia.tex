\begin{frame}{Problem ludologiczno-narracyjny w grach komputerowych}
Współczesne gry komputerowe są jedną z najważniejszych form mediów interaktywnych, a ich popularność rośnie zarówno w wąskich kręgach graczy, jak i w szerszym kontekście kulturowym. Przez te lata gry stopniowo zmieniały się z prostej rozrywki w wyrafinowane formy doświadczeń, które łączą w sobie różne aspekty: od grafiki, przez dźwięk, aż po fabułę. Prawdopodobnie najbardziej ekscytującą i złożoną kwestią, która pojawia się w analizach gier, jest to, co zostało nazwane problemem ludologiczno-narracyjnym. Pytanie brzmi, w jaki sposób mechanika gry - ludologia - łączy się z fabułą - narracją - w grach komputerowych i jakie wyzwanie stanowi to zarówno dla twórców gier, jak i badaczy.
\end{frame}
