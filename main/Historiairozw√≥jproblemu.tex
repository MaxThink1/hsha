\begin{frame}{Historia i rozwój problemu}
We wczesnych etapach rozwoju gier komputerowych (lata 70. i 80.) dominowały gry oparte na mechanice. Gracze skupiali się głównie na rywalizacji, zbieraniu punktów i pokonywaniu trudności w grze. W tym okresie narracja była zwykle uproszczona lub ograniczona do fabuły (np. „Space Invaders” lub „Pac-Man”).

Z czasem jednak, wraz z rozwojem technologii, gry zaczęły przybierać bardziej złożoną formę, zawierającą zarówno elementy mechaniczne, jak i fabularne. Wraz z pojawieniem się gier RPG, takich jak „Final Fantasy” i „The Elder Scrolls” w latach 90-tych, narracja stała się głównym czynnikiem wpływającym na doświadczenie gracza. W grach nie chodziło tylko o wyzwania i grywalność, ale także o opowiedzenie historii, w którą gracz mógł być zaangażowany. To połączenie grywalności i narracji zrodziło pytania o to, jak te dwa elementy mogą współistnieć.

Jednym z takich problemów jest połączenie ludologii i narratologii
Obecnie jedną z największych zagadek dla twórców gier jest znalezienie równowagi między mechaniką gry a opowiadaniem historii. Większość współczesnych gier równoważy te dwa aspekty, ale różne podejścia do łączenia grywalności z narracją prowadzą do różnych rezultatów.

Interaktywna narracja - w grach wideo takich jak „Wiedźmin 3” i „Red Dead Redemption 2” fabuła jest głęboko zintegrowana z rozgrywką. Gracz kontroluje wydarzenia w ramach opowieści i dokonuje wyborów, które wpływają na jej wynik. Taki tok myślenia daje graczowi poczucie kontroli nad światem danej gry i prowadzi do zmiany sposobu odczuwania gry i grania w nią, bardziej osobistego.

Fabuły z ograniczoną interakcją: Inne gry, takie jak „Uncharted” lub „The Last of Us”, mają bardziej zdefiniowaną fabułę, w której gracz niekoniecznie ma tak duży wpływ na wynik; niemniej jednak aspekt rozgrywki jest istotny. W tym przypadku narracja mocno wpisuje się w rozgrywkę, a fabuła działa jako tło dla działań gracza. Mimo to interakcja z fabułą jest ograniczona, co czasami prowadzi do wrażenia, że gracz jest bardziej biernym obserwatorem niż aktywnym uczestnikiem wydarzeń.

Wyzwania narracyjne w przypadku gier typu sandbox: W grach typu sandbox, takich jak „Minecraft” i „GTA V”, gracz trafia do otwartego świata bez zapewnionej fabuły; zamiast tego gracz sam tworzy swego rodzaju historię. Narracja jest w tym przypadku bardziej zróżnicowana i oparta na wyborach gracza; może to prowadzić do braku spójności w fabule, ale oferuje również możliwość większej swobody w tworzeniu własnej historii.
\end{frame}
